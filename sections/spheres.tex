\documentclass[../main.tex]{subfiles}

%\IfEq{\jobname}{\detokenize{main}}{}{%
%    \setcounterref{chapter}{../master-chap:hilbertpts}
%    \addtocounter{chapter}{+3}
%}


\begin{document} 

\section{A derived model for spheres}

\subsection{Foliations on spheres}

A transversely holomorphic foliation (THF) on a smooth manifold $M$ is a foliation $\cF$ of codimension $2n$ whose leaf space is equipped with the structure of a complex manifold \cite{Asuke,Rawnsley}. 
We will say that such a foliation $\cF$ equips $M$ with the a THF structure. 
%Suppose $M$ is a manifold equipped with a THF structure and let $\cF$ be the corresponding foliation of even codimension.
The product $X \times N$, where $X$ is a complex manifold and $N$ is a smooth manifold has a natural THF structure with $\cF$ the restriction of the tangent bundle of $N$ along the projection.
Locally, the THF structure is split of the form $\C^n \times \R^m$, whose coordinates we will denote by $(z_i ;  x_j)$.
The tangent space to the foliation $\cF$ is locally spanned by $\partial / \partial x_i$. 

Let $Q_{\R}$ be the (real) quotient bundle $\T_M / \T_\cF$. 
There is an integrable subbundle $V$ of the complexified tangent bundle $\T^\C_ M$ which is locally spanned by $\partial / \partial \zbar_i$ and $\partial / \partial x_j$. 
Integrability of $\cF$ ensures that $V$ is well-defined globally. 
Denote by $Q$ the (complex) quotient bundle $\T_\C M / V$.
Notice that there is a canonical isomorphism of complex bundles $Q \oplus \Bar{Q} = Q_\R \otimes_\R \C$. 

%Sections of $V$ form a Lie algebra with respect to the Lie bracket of vector fields. 

For each $p,q$ denote by $\cA^{p;q}$ smooth sections of the bundle $\wedge^p Q^\vee \otimes \wedge^q V^\vee$. 
The derivative along the leaves of the foliation defined by $V$ defines a map 
\[
\thfd \colon \cA^{p;q} \to \cA^{p;q+1}  .
\]
By integrability one has $\thfd^2 = \thfd \circ \thfd = 0$ and so $\thfd$ equips $\cA^{p;\bu} = \oplus_q \cA^{p;q}[-q]$ with the structure of a cochain complex for each $p$. 
The obvious exterior product $\cA^{p;q} \times \cA^{r;s} \to \cA^{p+r;q+s}$ further endows 
\[
\left(\cA^{\bu;\bu}, D\right) = \left(\oplus_p \cA^{p;\bu}[-p] , D \right) 
\]
with the structure of a bigraded commutative dg algebra.

Locally, a section of $\cA^{p;q}$ is of the form
\[
\alpha = \sum_{I,J,K} f_{I,J,K} (z,\zbar, x) \d^I z \d^J \zbar \d^K x 
\]
where the sum runs over multi indices $I = (i_1,\ldots,i_p)$, $J = (j_1,\ldots,j_r)$, $K = (k_1,\ldots,k_s)$ with $r + s = q$. 
In this formula we have used the shorthand notations 
\begin{align*}
\d^I z & = \d z_{i_1} \cdots \d z_{i_p} \\ 
\d^J \zbar & = \d \zbar_{j_1} \cdots \d \zbar_{j_r} \\
\d^K x & = \d x_{k_1} \cdots \d x_{k_s} .
\end{align*}
The differential $\thfd$ acting on the section $\alpha$ is 
\[
\thfd (\alpha) = \sum_{I,J,K} \left(\sum_{i} \frac{\del f_{I,J,K}}{\del \zbar_i} \d \zbar_i + \sum_{j} \frac{\del f_{I,J,K}}{\del x_j} \d x_j \right) \d^I z \d^J \zbar \d^K x .
\] 
Thus locally $\d'$ is simply a sum of the Dolbeault operator $\dbar$ on $\C^n$ and the de Rham operator $\d$ on $\R^m$.

\begin{dfn}
The \defterm{Dolbeault--de Rham cohomology} of $\wedge^p Q^\vee$ is
\[
H^{p,\bu}_{\thfd} (M) = H^{\bu} \left(\cA^{p;\bu}(M), \thfd \right) .
\]
We refer to $(\cA^{p;\bu}(M), \thfd)$ as the Dolbeault--de Rham complex of $\wedge^p Q^\vee$. 
\end{dfn}

More generally one can define the Dolbeault--de Rham complex of a vector bundle $E \to M$ compatible with the THF structure, but we will not need that. 

\parsec
Suppose that $M = X \times N$ is a split THF structure. 
Then there is an isomorphism of commutative dg algebras
\[
\left(\cA^{p;\bu}(M), \d'\right) \cong \left(\Omega^{p,\bu}(X) \hotimes \Omega^\bu(N) \, , \, \dbar \otimes \id + \id \otimes \d\right) 
\]
where $\Omega^{p,\bu}(X)$ is the usual Dolbeault complex of the complex manifold $X$ and $\Omega^\bu(N)$ is the de Rham complex of $N$.

\subsection{Cohomology of punctured $\C^n \times \R^m$}

Notice that any open submanifold of a THF manifold has the natural structure of a THF manifold. 
In particular, the submanifold 
\beqn \label{eqn:punctured}
M(n,m) \define (\C^n \times \R^m) \setminus 0 \subset \C^n \times \R^m 
\eeqn
has a THF structure.
We will characterize the Dolbeault--de Rham cohomology of~\eqref{eqn:punctured}.
Of course, this manifold is diffeomorphic to $S^{2n+m-1} \times \R_{>0}$, hence the full THF cohomology will have a deformation to the cohomology of the sphere $H^\bu(S^{2n+m-1})$.

\begin{thm}
Suppose that $n+m > 1$.
The cohomology
\beqn
H^{p;q} \define H^{p;q}_{\thfd}\left(M(n,m)\right)
\eeqn
admits the following description. 
\begin{itemize}
\item When $p = 0$ the cohomology is concentrated in degrees zero and $n+m-1$.
The cohomology $H^{0;0}$ is isomorphic to the algebra of holomorphic functions on $\C^n$.
The isomorphism is induced by the restriction of holomorphic functions along the composition
\[
\C^n \times \R^m \setminus 0 \hookrightarrow \C^n \times \R^m \twoheadrightarrow \C^n . 
\]
\item When $p = n$ the cohomology is
\end{itemize}
\end{thm}

\subsection{An algebraic model for the THF two-sphere}

Let $\sfR_{1,1}$ be the commutative algebra generated by variables 
\beqn
w,\lambda,x
\eeqn
subject to the relation
\beqn
w \lambda + x^2 = 1 .
\eeqn
Let $\sfA_{1,1}$ be the graded commutative algebra freely generated over $\sfR_{1,1}$ by a degree $+1$ element $\omega$.
Thus $\sfA_{1,1} = \sfR_{1,1} \oplus \sfR_{1,1} \omega [-1]$.
Define a differential
\beqn
\d' \colon \sfA_{1,1} \to \sfA_{1,1} [1]
\eeqn
on $\sfA_{1,1}$ by the formulas
\beqn
\d' (w) = 0, \quad \d' (\lambda) = - 2x \omega, \quad \d' (x) = w \omega .
\eeqn
Notice that the algebra of vector fields $\Vect^{poly}(\C) = \C[z] \del_z$ with polynomial coefficients naturally acts on $\sfA_{1,1}$ as a cochain complex.

\begin{prop}
There are $\Vect^{poly}(\C)$-equivariant isomorphisms
\beqn
H^0 (\sfA_{1,1}) \simeq \C[w] 
\eeqn
and
\beqn
H^1(\sfA_{1,1}) \simeq \C[\lambda] \omega 
\eeqn
\end{prop}

Consider the Dolbeault--de Rham complex 
\[
\cA^{\bu;(0)}(\C \times \R - \{0\})
\]
of the THF manifold $\C \times \R - \{0\}$ equipped with the Dolbeault--de Rham differential $\d'$.
Denote the coordinates on $\C \times \R$ by $(w,\wbar;t)$.

\begin{prop}
\label{prop:model1}
There is an injective morphism of commutative dg algebras 
\[
j \colon \sfA_{1,1} \hookrightarrow \cA^{\bu;(0)}(\C\times \R - \{0\}) .
\]
which is a dense embedding in cohomology.
On degree zero generators it is defined by $j(z) = z$, $j(\lambda) = \frac{\zbar}{r^2}$, $j(x) = \frac{t}{r}$ where $r^2 = z \zbar + t^2$.
On the degree one generator we define $j (\omega) = \omega_{S^2}$ where
\beqn
\omega_{S^2} \define \# \frac{2 \zbar \d t - t \d \zbar}{r^3} .
\eeqn
\end{prop}

\subsection{An algebraic model for the CR three-sphere}

Let $\sfR_{2,0}$ be the algebra generated by the degree zero variables 
\beqn
z_1,z_2,\lambda_1,\lambda_2
\eeqn
subject to the relation
\beqn
z_1 \lambda_1 + z_2 \lambda_2 = 1 .
\eeqn
Let $\sfA_{1,1}$ be the graded commutative algebra freely generated over $\sfR_{2,0}$ by a degree $+1$ element $\omega$.
Thus $\sfA_{2,0} = \sfR_{2,0} \oplus \sfR_{2,0} \omega [-1]$.

Define the linear map 
\beqn
\dbar \colon \sfA_{2,0} \to \sfA_{2,0}[1] 
\eeqn
by the rules
\begin{itemize}
	\item $\dbar(z_i) = 0$.
	\item $\dbar(\lambda_i) = \ep_{ij} z_j \omega$ ,
\end{itemize}
and extend its action to the entire graded algbra $\sfA_{2,0}$ by the rule that it is a graded derivation.
Notice that this differential is well-defined since
\beqn
\dbar(z_1 \lambda_1 + z_2 \lambda_2) = z_1 z_2 \omega + z_2 (-z_1 \omega) = 0 .
\eeqn

The cohomology $H^\bu(\sfA_{2,0}, \dbar)$ is concentrated in degrees zero and one. 
In degree zero, the cohomology is simply polynomials in the variables $z_1,z_2$:
\beqn
H^0(\sfA_{2,0},\dbar) \cong \C[z_1,z_2].
\eeqn
And in degree one, the cohomology can be identified with
\beqn
H^1(\sfA_{2,0},\dbar) \cong \C[\lambda_1,\lambda_2] \omega .
\eeqn

As a corollary of this result we have the following, see also \cite{FHK}.

\begin{prop}
\label{prop:modelA}
The dg algebra $(\sfA_{2,0},\dbar)$ is a model for punctured two-dimensional affine space
\beqn
\sfA \simeq \R \Gamma(\A^2 - \{0\}, \cO) .
\eeqn
\end{prop}

\begin{rmk}
The motivation for this model comes from the (algebraic) Dolbeault complex $\Omega^{0,\bu}_{alg}(\A^2 - \{0\})$ of punctured affine space.
Indeed, there is a quasi-isomorphism
\beqn
j \colon \sfA \hookrightarrow \Omega^{0,\bu}_{alg}(\A^2 - \{0\})
\eeqn
defined by $j(z_i) = z_i$, $j (\zbar_i / |z|^2) = \lambda_i$.
\end{rmk}

\begin{rmk} 
We point out that there is a formal version of $\sfA$ which we denote by $\Hat{\sfA}$.
The zeroth cohomology of this commutative dg algebra is $\C[[z_1,z_2]]$.
This is a model for the punctured formal two-dimensional disk.
In this model the vector spaces $H^0(\Hat{\sfA})$ and $H^1(\Hat{\sfA})$ are canonically dual using the residue pairing (see the next section).
\end{rmk}

\begin{rmk}
There is a standard \v{C}ech cover of $\A^2 - \{0\}$ consisting of two open sets $U_1 = \{z_1 \ne 0\}$ and $U_2 = \{z_2 \ne 0\}$.
The first \v{C}ech cohomology with respect to this cover can be identified with the $\C[z_1,z_2]$-module $z_1^{-1} z_2^{-1} \C[z_1^{-1},z_2^{-1}]$.
\end{rmk}



%\appendix
%
%\section{A model for punctured affine space}
%
%Consider the following algebra $\sfR_n$ on generators
%\beqn
%z_i, \lambda_i, \quad i=1,\ldots,n
%\eeqn
%subject to the relation
%\beqn
%\sum_{i=1}^n z_i \lambda_i = 1 .
%\eeqn
%
%Let $I$ denote a multi-index of length $|I|$
%\beqn
%I = (i_1,\ldots,i_{|I|})
%\eeqn
%where $i_j = 1,\ldots, n$.
%For each $k$ we introduce generators 
%\beqn
%\lambda_{I}, \quad |I|=k
%\eeqn
%of cohomological degree $k-1$
%which are totally skew-symmetric in the sense that
%\beqn
%\lambda_{\sigma(I)} = (-1)^\sigma \lambda_I .
%\eeqn
%
%Let $\sfA_n$ be the graded algebra over $\sfR_n$ generated by elements $\lambda_I$ for every multi-index of length $k=1,\ldots,n$.
%In other words
%\beqn
%\sfA_n = \sfR_n [\lambda_{i_{1} i_2}, \ldots, \lambda_{i_1 i_2 \cdots i_n}] .
%\eeqn
%
%\subsection{The two-dimensional case}
%
%The algebra $\sfA_2$ is generated by four degree zero elements 
%\beqn
%z_1,z_2,\lambda_1,\lambda_2
%\eeqn
%and a single degree one element
%\beqn
%\omega = \lambda_{12} = - \lambda_{21} .
%\eeqn
%These generators are subject to the single relation
%\beqn
%z_1 \lambda_1 + z_2 \lambda_2 = 1 .
%\eeqn
%
%In particular, $\sfA_2$ is concentrated in degrees zero and one.
%The differential is
%\beqn
%\dbar(z_i) = 0, \quad \dbar (\lambda_i) = - \ep_{ij} z_j \omega .
%\eeqn
%
%\subsection{The three-dimensional case}
%
%The algebra $\sfA_3$ is generated by six degree zero generators
%\beqn
%z_i, \lambda_i, \quad i=1,2,3
%\eeqn
%three degree one elements
%\beqn
%\lambda_{12} = - \lambda_{21}, \; \lambda_{23}=-\lambda_{32}, \; \lambda_{31} = - \lambda_{31}
%\eeqn
%and a single degree two element
%\beqn
%\omega = \lambda_{123} .
%\eeqn
%These generators are subject to the relations
%\beqn
%\sum^3_{i=1} z_i \lambda_i = 1
%\eeqn
%and
%\beqn
%\lambda_{ij} \lambda_{ik} = \pm \lambda_i \omega .
%\eeqn
%Notice that these relations imply that $\lambda_{ij} \omega=0$ \brian{does it?} and $\omega^2 = 0$.
%
%The complex is concentrated in degrees zero, one, and two.
%The differential is defined on degree zero generators by
%\beqn
%\dbar(z_i) = 0, \quad \dbar(\lambda_i) = \sum_{j \ne i} z_j \lambda_{ij} 
%\eeqn
%and on degree one generators by
%\beqn
%\dbar(\lambda_{ij}) = \pm \ep_{ijk} z_k \omega .
%\eeqn


\end{document}