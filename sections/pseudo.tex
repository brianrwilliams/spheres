\documentclass[../main.tex]{subfiles}

%\IfEq{\jobname}{\detokenize{main}}{}{%
%    \setcounterref{chapter}{../master-chap:hilbertpts}
%    \addtocounter{chapter}{+3}
%}


\begin{document} 

Let $\sfF(\hbar)$ be the Heisenberg vertex algebra defined over $\C[\hbar]$ generated by a single field $\phi$ and subject to the OPE
\beqn
\phi(z) \phi(w) \simeq \frac{\hbar}{(z-w)^2} .
\eeqn
Let $\sfF^{(N)} (\hbar) = \sfF(\hbar) \otimes_{\C[\hbar]} \cdots \otimes_{\C[\hbar]} \sfF(\hbar)$ be the $n$th fold tensor product of the Heisenberg vertex algebra whose fields we denote $\phi_1(z), \ldots, \phi_n(z)$.
Consider the fields $U^{(N)}_1(z), \ldots, U^{(N)}_N(z)$ in $\sfF^{(N)}(\hbar)$ defined by the formal expression
\beqn
: (\del_z + \phi_1(z)) \cdots (\del_z + \phi_N(z))\colon = \del_z^n + \sum_{k=1}^N U^{(N)}_i(z) \del_z^{n-k} .
\eeqn
The vertex algebra subalgebra of $\sfF^{(N)}(\hbar)$ generated by these fields is the vertex algebra $\sfW_N [\hbar]$. 
The specialization $\hbar=1$ is the standard $\sfW_N$ vertex algebra.
The classical limit $\sfW_N(\hbar) / (\hbar)$ is isomorphic to the Gelfand--Dickey Poisson vertex algebra of $N$th order differential operators.

For $M \leq N$ there is a vertex algebra morphism
\beqn
\sfW_M(\hbar) \to \sfW_N(\hbar) 
\eeqn
sending $U_i^{(M)}(z) \mapsto U_i^{(N)}(z)$ for $i \leq M$.
The inductive limit $\lim_{N \to \infty} \sfW_N(\hbar)$ is the vertex algebra $\sfW_{1+\infty} (\hbar)$.

\subsection{$\sfW_{1+\infty}$-algebras}

The vertex algebra $\sfW_{1+\infty,c}$ of central charge $c$ is a simple quotient of a vacuum module $\sfM_c$.
If $c \ne \Z$ then $\sfW_{1+\infty,c} = \sfM_c$. 

Linshaw constructs a two-parameter $W$-algebra $\sfW (c, \lambda)$ with fields of spin $(2,3,4,\ldots)$ of central charge $c$.
Tensoring with the Heisenberg algebra $\sfF$ we get another vertex algebra
\beqn
\sfV(c,\lambda) = \sfF \otimes \sfW(c,\lambda)
\eeqn
which has central charge $c+1$ with fields of spin $(1,2,3,\cdots)$.

Let $I \subset \C[c,\lambda]$ be the ideal generated by $4 \lambda (c-1) - 1$. 
Then the quotient vertex algebra
\beqn
\sfV(c,\lambda) / I
\eeqn
is defined over $\C[c,\lambda]/I$.

Let $D$ be the multiplicative set in the polynomial ring $\C[c,\lambda]$ generated by $c-1$.
Then consider the localized ring $R = D^{-1}\C[c,\lambda]/I $.
We can localize $\sfV(c,\lambda) / I$ to get a vertex algebra
\beqn
\sfV_{R,I} (c,\lambda) \define R \otimes_{\C[c,\lambda]/I} \otimes \sfV(c,\lambda) / I
\eeqn
defined over $R$.

As vertex algebras over $R$ one has
\beqn
\sfV_{R,I} (c,\lambda) \simeq R \otimes_{\C[c]} \cM_{c+1} .
\eeqn


\end{document}